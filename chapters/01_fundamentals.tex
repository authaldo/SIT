\chapter{Fundamentals}
\section{Motivation and Introduction}
The number of catalogued vulnerabilities is rising rapidly over time, ever since
IT existed. Security is all over the headlines both in dedicated news agencies
and in mainstream media. Ransomware attacks are popular recently and have highly
visible and critical targets such as hospitals and large companies. Running a
networked computer system with 100\% security is neither feasible nor possible.
But strong security is still achievable, and defending most attackers is
possible.

\section{Terminology}
When talking about ``secure'' systems, what is usually desired is a
``dependability'', meaning a system that shows no unexpected or unacceptable
behavior. A dependable system should fulfill multiple goals, including
\begin{itemize}
    \item Availability
    \item Reliability
    \item Safety
    \item Integrity
    \item Maintainability
    \item (Confidentiality)
\end{itemize}

The three major security goals are Confidentiality, Integrity and Availability
(\emph{CIA}). The main difference between dependability and security is that the
security is usually assessed from the viewpoint of a potential attack, while
dependability is considered in a more general context. Security can be seen as a
precondition of having a dependable system. A secure system is a system that can
achieve the three mentioned goals even facing an attacker.

\begin{description}[align=right,labelwidth=3cm]
    \item[Confidentiality] Protection of information against unauthorized access
    \item[Integrity] Protection against unauthorized change and destruction
    \item[Availability] Protection against rendering IT resources inaccessible
\end{description}

A \emph{threat} is defined by a potential error in the system, which could
enable an attacker to violate those objectives. A \emph{vulnerability} is a
concrete fault in the system that threaten one or more security objective. A
threat would be for example the possibility of a DDOS attack towards a web
service, a lack of resources to cope with the attack is a vulnerability. If an
attacker exploits the vulnerability, this is called an \emph{attack}.

The concepts of safety and security shall be differentiated.

When analyzing a system with regards to it security, two factors are considered:
The first is \emph{threat potential}, which estimates the likelihood of each
potential attack against the system. The second is the \emph{damage potential},
which asks what the impact to the system would be if an attack succeeds. Likely
high-impact attack scenarios can be mitigated by either reducing the impact of a
successful impact or by taking security measures reducing the likelihood of a
successful attack.

A few more useful definitions:
\begin{description}[align=right,labelwidth=3cm]
    \item[Identification] Assignment of an identifier
    \item[Authentication] Verification of an identity
    \item[Authorization] Assignment of permissions
    \item[Access Control] Protection of resources against unauthorized access
    \item[Privacy] Protecting personal information
\end{description}

\section{Attacks and Defenses}
\subsection{Attacks}
Attack can be categorized by many measures, like intention, approach or point of
attack.

Categories by intention can be:
\begin{description}
    \item[Denial of Service ]Making an IT system unabailable to users
    \item[Information Theft] Access to confidential information by unauthorized
          persons
    \item[Intrusion] Bypassing access control to gain access to a system
    \item[Tampering] Modification of stored or transmitted data
\end{description}

By approach:
\begin{itemize}
    \item Masquerading
    \item Eavesdropping
    \item Authorization Violation
    \item Loss/Modification
    \item Denial/Repudiation
    \item Forgery
    \item Sabotage
\end{itemize}

By point of attack:
\begin{itemize}
    \item Network
    \item Network services
    \item Operating system/Applications
    \item User
\end{itemize}

Another possible method of categorization is ``STRIDE'' categorization, which
stands for Spoofing, Tampering, Repudiation, Information disclosure, Denial of
Service and Elevation of privilege.

An attack often follows similar patterns. The first step is collecting
information of the system. This may expose a possible attack vector to the
attacker, which can then be used/tested. This is often repeated, as the first
attack is not necessarily successful. After a successful intrusion the next step
is often privilege escalation. This may allow the attacker to cover their
tracks, install back doors, etc. If the main goal of the attack is not reached
yet, this point may be used as the starting point for the next attack, towards
the initial goal.

Understanding and talking about attacks is vital when dealing with security, as
this is the very point we are trying to defend against.

\subsection{Security Mechanisms and Policies}
A Policy is a statement of what is, and what is not allowed. This allows the
distinction into ``authorized'' and ``unauthorized'' that we made already.

A Security mechanism is a method, tool or procedure that attempts to enforce
such policies. We can categorize those measures into prevention, detection and
recovery. It is possible to employ multiple measures, for example a gate as a
way of prevention, and a camera pointed at the gate which provides detection of
a successful attack.

\emph{Security through obscurity} or prohibiting reverse engineering and
attacking is not an alternative to real security.

Security mechanisms can never be judged by themselves. During risk assessment,
it is always necessary to evaluate the measures in the context in which they are
employed. The security of a system is determined by the weakest link in the
chain.
