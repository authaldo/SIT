\chapter{Embedded and Hardware Security}

Security in embedded systems is of special concern. Embedded systems are often
harder to patch, and often have direct real-world impact, sometimes in a safety
critical way. Non-mainstream operating systems are in use, and the CPU might not
offer all desired security features (see below). This results in a large attack
surface on a system that is not as well understood security-wise as desktop- or
server applications.

In the following, security will be considered from a lower level than before,
down to the hardware. Many of the security measures considered before can be
circumvented if lower-level access is present. Access control on a file system
for example is worthless if the attacker can connect the disk to another system
and dump all the contents. Software security is difficult if it relies on
libraries that can be exchanged by an attacker.

Placing security mechanisms at the lowest possible layer circumvents those
attacks.

\section{Introduction and x86 Privilege Levels}
At any point while an x86 processor is executing instructions, it is in some
\emph{privilege level} or \emph{protection ring}, usually indicated by a number
where higher numbers represent less privileged execution:

\begin{itemize}
    \item[(-1)] Virtual Machine Monitor, Hypervisor
    \item[0] Operating system kernel
    \item[1] Rest of the operating system
    \item[3] I/O Drivers etc.
    \item[4] Application software
\end{itemize}

Switching to a lower privilege level must be protected, while switching to a
higher level (less privileges) is usually easy. One important feature made
possible by this is protection of memory segments. Each memory segment contains
a Descriptor Privilege Level \textit{DPL}, and each process is assigned a
privilege level. If the Current Privilege Level \textit{CPL} > \textit{DPL}, the
CPU generates a protection fault and prevents access to memory accessible only
to lower privilege levels. This for example prohibits applications from
modifying operating system data structures.

Upgrading of the privilege level (setting it to a lower value) may happen for
example when a syscall is initiated and execution is passed to the operating
system. This provides a well defined ``gate'' to those privilege levels.

In order to prevent an application to misuse the privilege level of the syscall,
by for example instructing it to copy data into the process that the application
would otherwise not have access to, an additional \textit{Requested Privilege
Level} may be introduced (\emph{confused deputy problem}).

\section{Isolation and HW-based Attacks}
The privilege levels introduced above provide a means of isolation between the
operating and user programs, often called \textit{userspace} and
\textit{kernelspace}. Isolation is often advantageous to security as it provides
a barrier with defined interfaces between components. A ubiquitous form of
isolation is that between multiple processes on a single computer. Address
space, access rights and file descriptors are completely separate for multiple
processes. Special security-critical functions such as handling of private keys
may even be delegated to a dedicated hardware security module.

Attacks which circumvent those measures and leak data through channels other
than the predefined interfaces are called \emph{side-channel attacks}: Such
side-channels may be for example power consumption or electromagnetic emission
of a device. Countless such side-channels have been found, with varying degree
of real-world usage. Most mitigations against such attacks have been broken by
even cleverer side-channel attacks. Recent examples of side-channel attacks
exploit interference between adjacent DRAM cells in main memory or cachelines
that don't get evicted after branch prediction has failed.

\section{HW-based Security Mechanisms}
\subsection{Trusted Platform Module}
IDK, seems like something the copyright-industry would come up with...

\subsection{Physical Unclonable Functions}
See Rührmair et. al.