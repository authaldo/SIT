\chapter{Cryptography}
\section{Cryptographic Hash Functions and Random Numbers}
\subsection{Hash Functions}
A hash function is defined as a function $h: D \mapsto S$ with $|D| > |S|$. A
hash function is expected to fulfill more desired properties:
\begin{itemize}
    \item Compression: $|D| \gg |S|$
    \item Chaotic: Maximal change of the hash with minimal change in input
    \item Surjective: |S| is fully used
    \item Efficient calculation
\end{itemize}
Even with those properties, a hash function is not considered a cryptographic
hash function. Even a CRC fulfills those properties. The additional properties
of a \emph{cryptographic} hash functions are:

\paragraph{One way function:} Also called \emph{first pre-image resistance},
    this implies that given a hash, an input producing that hash can not be
    computed efficiently.

\paragraph{Weak collision resistance:} Also called \emph{second pre-image
    resistance}, implies that given a hash $h = \hash(m)$, an input $m' \neq m$
    with $\hash(m') = h$ can not be efficiently found.

\paragraph{(Strong) collision resistance:} No $m$ and $m' \neq m$ can be found
    efficiently with $\hash(m) \eq \hash(m')$. In contrast to weak collision
    resistance, a specific hash value is not required here.

Hash functions can be used in combination with a key in the form of
\emph{Message Integrity Code} and \emph{Message Authentication Code} to provide
an equivalent to digital signatures using symmetric cryptography. The hash based
MAC \emph{HMAC} is calculated as a hash over a combination of the message with a
key.

\subsection{Random Number Generators}
Pseudo-Random Number Generators \emph{PRNG}s which produce a deterministic
sequence of numbers from an initial seed are not suitable for cryptographic
applications such as key generation due to their predictability.
Non-Deterministic RNGs exist and often rely on external physical processes like
noise in electronic circuits. They do however often have a low data rate which
is not sufficient for all applications. A combination of a PRNG which is
(periodically) seeded by a true, non-deterministic RNG is an approach used in
practice.

Criteria for good RNGs are:
\begin{itemize}
    \item Statistical distribution as desired
    \item Independence: No repeating sequences, invariant to environmental
          conditions
    \item Speed of generation
    \item Long periodicity (PRNGs) / Non-reproducibility
\end{itemize}


\section{Encryption}
One important distinction between encryption schemes is the concept of symmetric
and asymmetric ciphers. Symmetric encryption is also called secret key
cryptography and relies on the presence of a secret key at both endpoints of the
communication. Asymmetric or public key encryption separates the key into a
public and private key for both parties. Information about the secret key is
never shared, and it can be used to decrypt messages which are encrypted using
the corresponding public key.

The algorithm itself is never considered secret, the security shall only depend
on the secret keys (Kerckhoffs principle).

Additional actors in a encryption scheme that are to be considered are the
passive (eavesdropping) attacker (\textit{Eve}), the active (man-in-the-middle)
attacker (\textit{Mallory}) and a trusted third party (\textit{Trent}).

\subsection{Symmetric Encryption}
Symmetrical encryption approaches usually follow the pattern of using basic
bitwise operations to form building blocks such as Feistel networks which are
then combined or repeated in multiple rounds. A challenge arises when the input
of the cipher is of variable length. Stream ciphers are suited to such a
problem, but the more common approach is to divide the input into blocks and use
a block cipher, operating with a fixed input size, multiple time. Simply
applying the same cipher with the same key to each block (\textit{electronic
code book}) however is not a good approach since repeated blocks in the input
will be directly reflected in the ciphertext. Alternatives include integrating
the previous ciphertext into the next block or including a counter in each
block, to avoid repeating inputs to the cipher altogether.

A standard block cipher used today is the \textit{Advanced Encryption Standard}
\emph{AES}, a block cipher with variants for different key lengths such as 128,
192 or 256 bit.

\subsection{Asymmetric Encryption}

Asymmetric encryption makes encryption possible even without a previously agreed
on private key.

\subsection{Diffie-Hellman Key Exchange}
The goal of the DIffie-Hellman key exchange algorithm is to establish a private
key between two users Alice and Bob, without compromising the secret key to an
attacker that has full visibility of the entire communication between Alice and
Bob.

\begin{algorithm}
    \SetAlgoLined
    
    \KwResult{private key $k$}
    choose public modulus $p$ (prime)\;
    choose public base $g$ (primitive root modulo $p$)\;
    \Begin(Alice){
        choose secret $a$\;
        $A = g^a \mod p$\;
        publish $A$;
    }
    \Begin(Bob){
        choose secret $b$\;
        $B = g^b \mod p$\;
        publish $B$\;
        $k = A^b \mod p$\;
    }
    \Begin(Alice){
        $k = B^a  \mod p$\;
    }
    \caption{Diffie-Hellman Key Exchange}
\end{algorithm}

This results in the same private key $k$ for both Alice and Bob. Generating this
private key requires knowledge of one of the secrets $a$ or $b$, which are only
known to the corresponding party.

Diffie-Hellman itself is not secure against \emph{Man-in-the-Middle Attacks}. If
an attacker \textit{Mallory} intercepts the communication, they could perform a
separate key exchange with both Alice and Bob. They would both have a shared key
with Mallory, unaware that the real communication partner is not Alice or Bob.
Mallory can then decrypt incoming messages, read or modify them, and re-encrypt
them using the second key.

\subsection{RSA}
RSA is an encryption scheme which allows encrypted communication without first
establishing a symmetric secret key (using Diffie-Hellman for example). Each
participant calculates both a public key and a secret key. The public key can be
used by any participant to encrypt messages, which can only be decoded using the
corresponding secret key, which is only known to the owner of the key.

\subsection{Digital Signatures}
If the integrity of a document and identity of the author are of concern, but
the contents are not necessarily encrypted, digital signatures are used. In
digital signatures, the signature is generated for a specific message (usually a
hash of the document) with the private key of the author. The verification of
the signature is possible using the corresponding public key and again the
message. This is similar to the reverse of the public key encryption scheme.

\subsection{Strength of Cryptographic Approaches}
The security of cryptographic algorithms can be categorized in the following
categories:

\begin{itemize}
    \item \textbf{Empirically secure:} The approach is secure because no attacks
    against it have been successful, and analysis has not found a specific
    weakness
    \item \textbf{(Formally) proven secure:} The encryption is proven to be a
    mathematically hard problem
    \item \textbf{Unconditionally secure:} ``An attacker cannot extract any
    information from the encrypted text''
\end{itemize}

The only unconditionally secure approach is the \emph{one-time pad}, where the
key and message have the same length and a unique, random key is used for every
transmission. Since the key has no inherent correlation, the probability of
recovering any arbitrary plaintext is identical, which makes even brute force
attacks impossible.
