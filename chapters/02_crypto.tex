\chapter{Cryptography}
\section{Cryptographic Hash Functions and Random Numbers}
\subsection{Hash Functions}
A hash function is defined as a function $h: D \mapsto S$ with $|D| > |S|$. A
hash function is expected to fulfill more desired properties:
\begin{itemize}
    \item Compression: $|D| \gg |S|$
    \item Chaotic: Maximal change of the hash with minimal change in input
    \item Surjective: |S| is fully used
    \item Efficient calculation
\end{itemize}
Even with those properties, a hash function is not considered a cryptographic
hash function. Even a CRC fulfills those properties. The additional properties
of a \emph{cryptographic} hash functions are:

\paragraph{One way function:} Also called \emph{first pre-image resistance},
    this implies that given a hash, an input producing that hash can not be
    computed efficiently.

\paragraph{Weak collision resistance:} Also called \emph{second pre-image
    resistance}, implies that given a hash $h = \hash(m)$, an input $m' \neq m$
    with $\hash(m') = h$ can not be efficiently found.

\paragraph{(Strong) collision resistance:} No $m$ and $m' \neq m$ can be found
    efficiently with $\hash(m) \neq \hash(m')$. In contrast to weak collision
    resistance, a specific hash value is not required here.

Hash functions can be used in combination with a key in the form of
\emph{Message Integrity Code} and \emph{Message Authentication Code} to provide
an equivalent to digital signatures using symmetric cryptography. The hash based
MAC \emph{HMAC} is calculated as a hash over a combination of the message with a
key.

\subsection{Random Number Generators}
Pseudo-Random Number Generators \emph{PRNG}s which produce a deterministic
sequence of numbers from an initial seed are not suitable for cryptographic
applications such as key generation due to their predictability.
Non-Deterministic RNGs exist and often rely on external physical processes like
noise in electronic circuits. They do however often have a low data rate which
is not sufficient for all applications. A combination of a PRNG which is
(periodically) seeded by a true, non-deterministic RNG is an approach used in
practice.

Criteria for good RNGs are:
\begin{itemize}
    \item Statistical distribution as desired
    \item Independence: No repeating sequences, invariant to environmental
          conditions
    \item Speed of generation
    \item Long periodicity (PRNGs) / Non-reproducibility
\end{itemize}


\section{Encryption}
One important distinction between encryption schemes is the concept of symmetric
and asymmetric ciphers. Symmetric encryption is also called secret key
cryptography and relies on the presence of a secret key at both endpoints of the
communication. Asymmetric or public key encryption separates the key into a
public and private key for both parties. Information about the secret key is
never shared, and it can be used to decrypt messages which are encrypted using
the corresponding public key.

The algorithm itself is never considered secret, the security shall only depend
on the secret keys (Kerckhoffs principle).

Additional actors in a encryption scheme that are to be considered are the
passive (eavesdropping) attacker (\textit{Eve}), the active (man-in-the-middle)
attacker (\textit{Mallory}) and a trusted third party (\textit{Trent}).

\subsection{Symmetric Encryption}

