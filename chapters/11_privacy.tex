\chapter{Data Protection and Privacy}
\section{Privacy Motivation}
\todo[inline]{privacy motivation}

\section{Privacy by Design and PETs}
\subsection{Anonymous Communication: TOR}
TOR implements a form of onion routing and a mix-network: A message for a server
is encrypted with its public key. It is however not directly sent to this
server, but through another server, which requires an additional layer of
encryption for the intermediate server. (In practice, this asymmetric encryption
may be replaced by a key exchange and symmetric encryption for the actual data,
for performance reasons). This is repeated such that there are three
intermediate \textit{onion routers} between the source and destination. This
way, none of the routers sees both the source and destination: The first router
knows about the source, the last router knows about the destination, the middle
router does not know the source or destination.

Attacks are possible by correlation of packets inside the network, incorrect
usage of DNS or identification by the message payload.

\subsection{Blind Signatures}
A blind signature is needed when a document shall be signed without revealing it
to the signing party. This works by first ``blinding'' the message, signing the
blinded message, then unblinding the signed, blinded message. This results in a
signed form of the original message.

These kind of blind signatures are useful for electronic cash, which should
provide anonymity, verifiable authenticity and protection against double
spending. An analogy works as following: The user who wants to spend money puts
an empty piece of paper with carbon paper in a sealed envelope. The bank signs
that envelope on the outside, which makes the envelope with the paper inside
worth 1\$. While signing the envelope, the signature was printed on the paper
inside the envelope. This signed paper is now used as payment. The bank has
however not seen the actual paper inside the envelope. Double spending is
prevented by containing a serial number on the paper, which the merchant
verifies with the bank before accepting payment.

\subsection{Group Signatures}
A group of users has a shared public key, but each user has an individual
private key. A message should now be signed with a private key, and verification
should be possible via the group public key. Group anonymity is provided since
it is only confirmed during validation that a member of the group produced the
signature, but not which exact member.

A problem is that the group manager, who issues keys to the members, can spoof
any participants identity and acts as a kind of trusted third party. The group
manager might also be able to identify the individual member from a signature.

\subsection{Attribute-Based Credentials}
Using \emph{Attribute Authentication}, the user can provide proof of an
attribute without revealing their own identity. Multiple authentications should
not be linkable. Examples could include public transport tickets, where the
owner should only be required to prove authorization to travel on the current
route, without revealing identity or entire travel plans. ABCs can also
implement pseudonymous or not-at-all-anonymous authentication.

In an ABC scenario, the ABC authority first authorizes an issuer, and issues a
device (smartcard) to the user. The issuer can then issue credentials containing
attributes stored on the card. The card can then generate a selective proof
revealing some attribute to a relying party.
