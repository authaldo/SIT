\chapter{OS Security}
\textit{Operating system-} and \emph{host-security} has the goals of protecting
stored data, running processes and the operating system itself. This
necessitates some form of authentication, to distinguish between users on a
multi-user system, as well as an access control system which assigns permissions
to users. A useful concept applied here is isolation, which is applied to users
and processes. Raw hardware access is usually restricted, and only allowed to
privileged code within the operating system. Protection against external access
may also be part of host security.

\section{Concepts and Reference Monitors}
A \emph{reference monitor} is an abstract machine which mediates all accesses to
objects by subjects (see \cref{chapter:access_control}). Reference monitors can
be implement on any level of the system. Reference monitors in the systems
hardware include the MMU and privileged execution modes. Kernel level reference
monitors are for example implemented in the file system or capability system.
Applications can also contain reference monitors, which may be the case in web
server applications, or run completely inside another program which implements a
reference monitor such as the Java virtual machine or a database engine.

The \emph{security kernel} is the hardware, firmware and software of a trusted
computing base which implements the reference monitor. It must mediate all
accesses, be protected from modification and be verifiable as correct.

The \emph{trusted computing base (TCB)} is the totality of protection mechanisms
within a computer system. This includes hardware, firmware and software which is
responsible for enforcing a security policy. The enforcement of the security
policy must only depend on the TCB, the rest of the operating system need not to
be trusted.

\section{Virtualization and Mandatory Access Control}
\section{Use-Case: iOS Security}
