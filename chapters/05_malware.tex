\chapter{Malware}
\section{Buffer Overflow Attacks}
Buffer overflows are one of the main security vulnerabilities found today.
Modern mitigations make them not as trivial in practice as in theory. The most
common form is a \emph{stack overflow}. The goal is to overwrite the return
address of the current stack frame, usually by exploiting unchecked array-access
which allows writing to the address where the return address is located. The
attacker could instruct the program to return to a different part of the
program, for example skipping authentication checks. If write access to an
executable page is also present, the attacker could also first inject code and
then jump to that code, executing arbitrary instructions. If the place of the
injected shellcode is not known exactly, one technique is to insert \texttt{NOP}
instructions before the actual exploit (\textit{NOP sled}).

Mitigations include canaries, which are values placed between the variables and
return address on the stack to indicate buffer overflows. Injection of shellcode
is usually protected against by not allowing memory pages with both write and
execute permissions. Type-safe languages also offer features making access to
memory outside of variables impossible. Address space layout randomization makes
guessing memory locations more difficult.


\section{Introduction to Malware}
Malware is a generic term for software, which is designed to perform a function
undesirable or harmful to the user. Categorization of malware is possible by the
approach of spreading:
\begin{itemize}
    \item A \emph{virus} is a program which spreads by abusing other (harmless)
          programs
    \item A \emph{worm} spreads autonomously over a network
    \item A \emph{trojan} disguises itself as a harmless program
\end{itemize}

Especially malware that spreads over the network can be further categorized, the
main distinction being the amount of interaction a user has to perform in order
to be infected, which can lead from downloading and opening an attachment to
only visiting a website or even zero-interaction vulnerabilities.

Initial infection may also happen completely offline, for example via the ``lost
USB stick'' tactic in a more targeted attack.

The malicious payload can assume many forms and achieve a multitude of goals.
Some examples of payload functionality include:
\begin{itemize}
    \item Deleting data
    \item Spying, exfiltrating data
    \item Enabling remote access
    \item Denial Of Service (\textit{DOS})
    \item Physical damage
    \item Encryption of data (and demanding ransom)
    \item Abusing resources (crypto mining)
\end{itemize}

More often than not, those are motivated by financial gains, which is apparent
with the recent waves of large scale ransomware attacks.

Malware may also be categorized into mass infection and targeted attacks
(Advanced Persistent Threat \textit{APT}). While the former is focused entirely
on maximizing the number of infected hosts, the latter is focused on a specific
target.


\section{Botnets and Targeted Attacks}

Botnets are established with no particular payload. The initial infection
happens with a \textit{dropper}, which connects to a command and control server.
The actual payload is then downloaded from this server, which allows the botnet
to be rented out to perform a number of different attacks that all benefit from
a large amount of infected machines.
